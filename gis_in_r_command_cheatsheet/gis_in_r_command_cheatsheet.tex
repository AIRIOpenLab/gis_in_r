\documentclass[10pt]{article} 
\usepackage{amsfonts, amsmath, amssymb} 
\usepackage{dcolumn, multirow} 
\usepackage{setspace} 
\usepackage{epsfig, subfigure, subfloat, graphicx}
\usepackage{tabularx} 
\usepackage{anysize, indentfirst, setspace} 
\usepackage{verbatim, rotating, paralist}
\usepackage{latexsym} 
\usepackage{amsthm} 
\usepackage{fullpage} 
\usepackage{longtable} 
\usepackage{graphicx} 
\usepackage{mathabx} 
\usepackage{txfonts} 
\usepackage{amsfonts} 
\usepackage{parskip} 
\usepackage{stmaryrd} 
\usepackage{mathrsfs} 
\usepackage{dsfont} 
\usepackage{comment} 
\usepackage{url} 
\usepackage{rotating} 
\usepackage{appendix}
\usepackage{natbib} 
\usepackage{tablefootnote}
\usepackage{hyperref}
\usepackage{multicol}
\setlength{\columnsep}{1cm}
\usepackage{enumitem}
\setlist[itemize]{noitemsep, topsep=0pt}
\setlist[enumerate]{noitemsep, topsep=0pt}
\usepackage[margin=2.5cm]{geometry}

\title{GIS in R Command Cheat Sheet}
\date{\today}


\begin{document} 
\maketitle
% \begin{multicols}{2}
	
\section*{Vector Data}
\centerline{\textbf{Installing}}
Update R to version $>$ 3.1.
On Windows:
\begin{itemize}
	\item \texttt{install.packages(c(``sp'',``raster''))}
	\item \texttt{install.packages(``rgdal'')}
\end{itemize}
On OSX:
	\begin{itemize}
		\item \texttt{install.packages(c(``sp'',``raster''))}
		\item Download and install 	\href{http://www.kyngchaos.com/files/software/frameworks/GDAL_Complete-1.11.dmg}{\underline{GDAL Complete}}
		\item Download \href{http://www.kyngchaos.com/files/software/frameworks/rgdal-0.9.1-1.dmg}{\underline{rgdal}} package. 
		\item Open .dmg file and place \texttt{rgdal\_0.9-1.tgz} on desktop.
		\item Run \texttt{install.packages("$\sim$/Desktop/rgdal\_0.9-1.tgz",repos=NULL)}
	\end{itemize}

\hrulefill \\ 
\centerline{\textbf{Creating Spatial Objects}} \\
\underline{\textbf{POINTS:}}\\
\textbf{Points}: \texttt{SpatialPoints([ matrix of coordinates] )}\\
\textbf{Points with DF}: \texttt{SpatialPointsDataFrame([Spatial Points Obj] , [ DataFrame] )}\\
\\
\underline{\textbf{POLYGONS:}}\\
\textbf{Polygon}: \texttt{Polygon([matrix of coordinates of vertices])}\\
\textbf{Collection of Polygons}: \texttt{Polygons([list of Polygon Objs], [names for Polygons])}\\
\textbf{Collection of SPATIAL Polygons}: \texttt{SpatialPolygons([list of Polygon\textbf{s} Objs], [names for Polygons])}
\begin{itemize}
	\item \emph{Spatial Polygons are Polygons with associated projection data}
\end{itemize}
\textbf{Spatial Polygons with DF}:\texttt{SpatialPolygonsDataFrame([SpatialPolygons Obj, dataframe])}

\hrulefill \\ 
\centerline{\textbf{Loading Spatial Objects from Files}} \\

\[ fill in \]

\hrulefill \\ 
\centerline{\textbf{Interrogating Spatial Objects}} \\
\textbf{Quick summary:} \texttt{summary([spatial\_object])}\\
\textbf{Longer summary of contents:} \texttt{str([spatial\_object])}\\
\textbf{Full list of contents:} \texttt{attributes([spatial\_object])}\\
\textbf{Check if projected:} \texttt{is.projected([spatial\_object])}\\

\underline{\textbf{EXTRACT ATTRIBUTES:} }\\
\textbf{Bounding Box:} \texttt{bbox([spatial\_object])}\\
\textbf{Get full projection info:} \texttt{proj4string([spatial\_object])}\\
\textbf{Get associated coordinates:} \texttt{coordinates([spatial\_object])}\\


\section*{Raster Data}

%\end{multicols}



\end{document}
