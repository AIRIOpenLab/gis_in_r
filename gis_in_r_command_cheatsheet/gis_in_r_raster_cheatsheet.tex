\documentclass[10pt]{article} 
\usepackage{amsfonts, amsmath, amssymb} 
\usepackage{dcolumn, multirow} 
\usepackage{setspace} 
\usepackage{epsfig, subfigure, subfloat, graphicx}
\usepackage{anysize, indentfirst, setspace} 
\usepackage{verbatim, rotating, paralist}
\usepackage{latexsym} 
\usepackage{amsthm} 
\usepackage{fullpage} 
\usepackage{graphicx} 
\usepackage{amsfonts} 
\usepackage{dsfont} 
\usepackage{comment} 
\usepackage{hyperref}
\usepackage{multicol}
\usepackage{txfonts} 
\usepackage{parskip} 

\setlength{\columnsep}{1cm}
\usepackage{enumitem}
\setlist{nosep}

\usepackage[margin=2.5cm]{geometry}

\title{GIS in R Command Cheat Sheet \\ \emph{Raster Data}}
\date{\today}
\author{Nick Eubank}

\begin{document} 
\maketitle
% \begin{multicols}{2}

%%%%%%%%%%
% Installing
%%%%%%%%%%
	
\centerline{\textbf{Installation of Relevant Packages}}
\textbf{Packages:}
\begin{itemize}
	\item \texttt{raster}: tools for raster datasets
\end{itemize}

\section*{Building and Managing Raster Objects}

%%%%%%%%%%
% Creating from Scratch
%%%%%%%%%%

\centerline{\textbf{Creating Rasters From Scratch}} 

\underline{\textbf{RasterLayer (the skeleton):}}\\
\hspace*{0.3cm}\texttt{newRL <- raster(ncol=10, nrow=20, xmn=0, xmx=10,ymn=-10,ymx=10)}\\
\underline{\textbf{RasterLayer w/ Data (skeleton + data):}}\\
\hspace*{0.3cm}\texttt{values(newRL) <- [vector]}
\begin{itemize}
	\item Length of vector should match total number of cells in Raster Layer obj
	\item vector entries associated with raster cells in order, with top left cell as \emph{1}, increasing left to right, then top to bottom, ending with bottom right cell. 
\end{itemize}


%%%%%%%%%%
% Load from Files 
%%%%%%%%%%

\hrulefill \\ 
\centerline{\textbf{Loading Spatial Objects from Files}} \\
\texttt{dem <- raster("file name.fileextension")}
\begin{itemize}
	\item Pass the entire filename  -- path, filename, and extension -- unlike in \texttt{readOGR()}.
\end{itemize}

%%%%%%%%%%
% Interrogating
%%%%%%%%%%

\hrulefill \\ 
\centerline{\textbf{Interrogating Raster and Setting Values}} \\
\textbf{Quick summary:} just type name of raster object \\
\textbf{Check if has values:} \texttt{hasValues(Raster obj)}\\

\textbf{Viewing or Setting Values:} In general, \texttt{raster} commands will return a value if just typed, and will set a value if an assignment is made. So \texttt{nrow(Raster obj)} gets number of rows, \texttt{nrow(Raster obj)<-5} sets number of rows to 5. Among these:
\begin{itemize}
	\item \textbf{Number of rows, columns, resolution:} \texttt{nrow(Raster obj)},\texttt{ncol(Raster obj)},\texttt{res(Raster obj)}
	\item \textbf{Values:} \texttt{values(Raster obj)}\\
\end{itemize}

%%%%%%%%%%
% Projections
%%%%%%%%%%

\hrulefill \\ 
\centerline{\textbf{Managing Projections}} \\
\emph{Note: similar to vector data, but without the intermediate CRS() step -- just pass the proj4 string.}\\
\textbf{Assigning projection by EPSG code:} \texttt{projection([Raster obj]) <-"+init=EPSG:4326"}\\
\textbf{Get projection from Spatial obj:} \texttt{projection([Raster obj])} \\
\textbf{Re-project:} \\
\hspace*{0.3cm}\texttt{reprojectedRaster <- projectRaster([raster obj],crs=[proj4 string for new projection])}\\
\href{http://www.spatialreference.org/}{\underline{\textbf{Projection code database}}} \\




\end{document}
